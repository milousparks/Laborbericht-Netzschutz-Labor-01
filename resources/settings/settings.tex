%Dokumentenklasse "scrbook" - Erweitert um den Verweis auf die Verzeichnisse und Texteigenschaften
\documentclass[12pt, a4paper, oneside, parskip=half, listof=totoc, bibliography=totoc, numbers=noendperiod]{scrreprt}


%Anpassung der Seitenränder (Standard bottom ca. 52mm anbzüglich von ca. 4mm für die nach oben rechts gewanderte Seitenzahl)
\usepackage[bottom=48mm,left=25mm,right=25mm]{geometry}

%Tweaks für scrbook
\usepackage{scrhack}

%Blindtext
\usepackage{blindtext}

%Erlaubt unteranderem Umbrücke captions
\usepackage{caption}

%Stichwortverzeichnis
\usepackage{imakeidx}

%Kompakte Listen
\usepackage{paralist}

%Zitate besser formatieren und darstellen
\usepackage{epigraph}
\usepackage{enumitem}
%Tabeln
\usepackage{booktabs}

%Glossar, Stichworverzeichnis (Akronyme werden als eigene Liste aufgeführt)
\usepackage[toc, acronym]{glossaries} 
\usepackage{physics}
%Anpassung von Kopf- und Fußzeile
%beinflusst die erste Seite des Kapitels
\usepackage[automark,headsepline]{scrlayer-scrpage}
\input{resources/styles/header_footer}

%Auskommentieren für die Verkleinerung des vertikalen Abstandes eines neuen Kapitels
%\renewcommand*{\chapterheadstartvskip}{\vspace*{.25\baselineskip}}

%Zeilenabstand 1,5
\usepackage[onehalfspacing]{setspace}

%Verbesserte Darstellung der Buchstaben zueinander
\usepackage[stretch=10]{microtype}

%Deutsche Bezeichnungen für angezeigte Namen (z.B. Innhaltsverzeichnis etc.)
\usepackage[ngerman]{babel}

%Unterstützung von Umlauten und anderen Sonderzeichen (UTF-8)
\usepackage{lmodern}
\usepackage[utf8]{luainputenc}
\usepackage[T1]{fontenc}

%Einfachere Zitate
\usepackage{epigraph}

%Unterstützung der H positionierung (keine automatische Verschiebung eingefügter Elemente)
\usepackage{float} 

%Erlaubt Umbrüche innerhalb von Tabellen
\usepackage{tabularx}

%Erlaubt Seitenumbrüche innerhalb von Tabellen
\usepackage{longtable}

%Erlaubt die Darstellung von Sourcecode mit Highlighting
\usepackage{listings}

%Definierung eigener Farben bei nutzung eines selbst vergebene Namens
\usepackage[table,xcdraw]{xcolor}

%Tikz für Statemaschine drawing
\usepackage{tikz}
\usetikzlibrary{automata, positioning, arrows,arrows.meta,shapes.geometric}
\tikzset{
	->, %makes the edges directed
	>=stealth', % makes the arrow heads bold
	node distance=3cm, % specifies the minimum distance between two nodes. Change if necessary.
	every state/.style={thick, fill=gray!10}, % sets the properties for each ’state’ node
	initial text=$ $, % sets the text that appears on the start arrow	
}
\tikzstyle{startstop} = [rectangle, rounded corners, minimum width=3cm, minimum height=1cm,text centered, draw=black, fill=red!30]
\tikzstyle{io} = [trapezium, trapezium left angle=70, trapezium right angle=110, minimum width=3cm, minimum height=1cm, text centered, draw=black, fill=blue!30]
\tikzstyle{process} = [rectangle, minimum width=3cm, minimum height=1cm, text centered, draw=black, fill=orange!30]
\tikzstyle{decision} = [diamond, minimum width=3cm, minimum height=1cm, text centered, draw=black, fill=green!30]
\tikzstyle{arrow} = [thick,->,>=stealth]
%Grafiken (wie jpg, png, etc.)
\usepackage{graphicx}
\usepackage{makecell}
\usepackage{siunitx}
\usepackage{mathtools}

%Grafiken von Text umlaufen lassen
\usepackage{wrapfig}

%Ermöglicht Verknüpfungen innerhalb des Dokumentes (e.g. for PDF), Links werden durch "hidelink" nicht explizit hervorgehoben
\usepackage[hidelinks,german]{hyperref}
\usepackage{cleveref}


%Einbindung und Verwaltung von Literaturverzeichnissen
\usepackage{csquotes} %wird von biber benötigt
\usepackage[style=alphabetic, backend=biber, bibencoding=ascii]{biblatex}
\addbibresource{references/references.bib}